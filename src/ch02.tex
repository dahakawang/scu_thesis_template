\chapter{如何使用}
\section{使用环境}
本模板需要依赖于\LaTeXe 、Xe\TeX 以及C\TeX ,因此在你使用前请确保这些发行版已经安装妥当。本文件全部使用UTF-8编码,并使用Xe\LaTeX 编译,以支持国际化和TrueType技术字体。

\section{文件结构}
本模板由以下文件构成:
\begin{itemize}
\item \textcolor{blue}{main.tex} - \LaTeX 基本框架,你可以在此添加你需要的
  Package
\item \textcolor{blue}{make.bat} - 运行在Windows上的编译脚本,双击即可执行,他可以免去你敲代码编译
  的麻烦 \smiley
\item \textcolor{blue}{Makefile} - 运行在Linux上的编译脚本,使用 \verb|make| 命令完成
  编译
\item \textcolor{blue}{scuthesis.sty} - 川大毕设论文格式样式包,你不需要了解这个文件(除非本模板板式不合符你的需求)
\item \textcolor{blue}{src/basic\_info.tex} - 定义论文作者基本信息
\item \textcolor{blue}{src/prologue.tex} - 包含了封面、中英文摘要以及目录的定义
\item \textcolor{blue}{src/epilogue.tex} - 包含了论文参考文献、附录等信息
\item \textcolor{blue}{src/ch*.tex} - 论文每一章具体内容
\item \textcolor{blue}{ref/refs.bib} - bibtex文献库,推荐使用 \href{http://jabref.sourceforge.net/}{JabRef} 维护
\end{itemize}

\section{如何使用}
首先说明一下,本教程是一篇self-contained的文章,本文章是直接编译本\LaTeX 模板得到,你可以具体参考模板源代码内容以学习如何使用。但是为了阐明脉络, 下面我将以一次完整使用的形式展示如何使用本模板。
\subsection{一个三级标题}
首先,你需要填写自己的基本信息,例如姓名、学号之类,你需要打开basic\_info.tex文件将其填写进去。然后你需要书写你的摘要,在prologue.tex里面是中英文摘要的定义处,你可以在其中编写摘要。假设你是\LaTeX 的老用户,你可能需要自己包含一些Package,那么你可以在main.tex中添加usepackage命令。
\subsubsection{四级标题}
另外,随着章节数目增多,你可以自行新建chxx.tex文件,并将其在main.tex中用include指令包含进来。最后,为了编译你的论文,你需要使用xelatex命令。不过目前Windows用户可以直接双击make.bat生成论文。

\section{问题}
\begin{enumerate}
\item 中英文摘要分别不能超过一页,否则第二页的板式会有问题(由于本人精力有限,且该问题出现几率较小,目前暂未打算修复这个问题)。
\item 所有文件必须是UTF-8编码,否则编译不能通过。
\end{enumerate}